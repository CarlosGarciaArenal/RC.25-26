%%***************************************************
\documentclass[a4paper,12pt]{article}

%% PACKAGES
\usepackage{amssymb}% to have some extra types of fonts
\usepackage{amsmath}
\usepackage{enumitem}
%% - writing in Spanish
\usepackage[latin1]{inputenc}% To type Spanish accents
\usepackage[spanish]{babel}% to have Spanish captions
%% - doc. formatting
\usepackage[a4paper,left=3.1cm,right=3.1cm,bottom=2.5cm,top=2.5cm]{geometry}
\usepackage{fancyhdr}
\pagestyle{fancy}
\fancyhf{}

\fancyhead[L]{Universidad de Cantabria}    % L = Left
\fancyhead[C]{Grado en Matemáticas}       % C = Center
\fancyhead[R]{Grado en Ing. Informática}      % R = Right
\fancyfoot[R]{\thepage}

\title{Representación del conocimiento \\ Práctica 1}
\author{Víctor Castañeda Balmori, Mario Cuesta Rivavelarde, \\
Carlos García Arenal, Laro Ayesa Sánchez y Alexandru Solovei Popa}
\date{03/10/2025}

\begin{document}

\maketitle
\thispagestyle{fancy}
El objetivo de la práctica es programar el algoritmo que determina si dos nodos de un grafo son separables dado otro conjunto de nodos. Es decir, el algoritmo que determina si $X \perp _g Y \mid Z$. En este documento analizamos la complejidad del código implementado en el fichero \textit{practica1.py}. \\ \\
Para el desarrollo de la práctica hemos dividido el trabajo en el grupo en los 3 pasos del algoritmo. El análisis de la complejidad lo hemos hecho, por tanto, primero por separado cada función (o cada paso con sus funciones) y después en conjunto.

\section{Complejidad del paso 1}

\section{Complejidad del paso 2}

\section{Complejidad del paso 3}





\end{document}
