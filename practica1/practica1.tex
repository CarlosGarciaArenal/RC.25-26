%%***************************************************
\documentclass[a4paper,12pt]{article}

%% PACKAGES
\usepackage{amssymb}% to have some extra types of fonts
\usepackage{amsmath}
\usepackage{enumitem}
%% - writing in Spanish
\usepackage[latin1]{inputenc}% To type Spanish accents
\usepackage[spanish]{babel}% to have Spanish captions
%% - doc. formatting
\usepackage[a4paper,left=3.1cm,right=3.1cm,bottom=2.5cm,top=2.5cm]{geometry}
\usepackage{fancyhdr}
\pagestyle{fancy}
\fancyhf{}

\fancyhead[L]{Universidad de Cantabria}    % L = Left
\fancyhead[C]{Grado en Matemáticas}       % C = Center
\fancyhead[R]{Grado en Ing. Informática}      % R = Right
\fancyfoot[R]{\thepage}

\title{Representación del conocimiento \\ Práctica 1}
\author{Víctor Castañeda Balmori, Mario Cuesta Rivavelarde, \\
Carlos García Arenal, Laro Ayesa Sánchez y Alexandru Solovei Popa}
\date{03/10/2025}

\begin{document}

\maketitle
\thispagestyle{fancy}
El objetivo de la práctica es programar el algoritmo que determina si dos nodos de un grafo son separables dado otro conjunto de nodos. Es decir, el algoritmo que determina si $X \perp _g Y \mid Z$. En este documento analizamos la complejidad del código implementado en el fichero \textit{practica1.py}. \\ \\
Para el desarrollo de la práctica hemos dividido el trabajo en el grupo en los 3 pasos del algoritmo. El análisis de la complejidad lo hemos hecho, por tanto, primero por separado cada función (o cada paso con sus funciones) y después en conjunto.

\section{Complejidad del paso 1}

El paso 1 consiste en identificar aquellos nodos que sean nodos hoja, y eliminarlos. Por lo tanto, se recorren todos los nodos (\textbf{n}) y se mantiene una lista con aquellos nodos que aún deban explorarse. Al explorar un nodo, si se elimina de la lista, se vuelven a añadir sus predecesores. En caso contrario, no sucede nada. \\ \\
Así, el peor caso será aquel en que los nodos $\in X \cup Y \cup Z$ (que no pueden ser eliminados) no tengan ningún predecesor y el primer nodo que se pueda eliminar sea el último en ser explorado. En este caso, se explorarán todos los nodos ($O(n)$), y después se irán explorando de nuevo para ir eliminándolos ($O(n)$), resultando un grafo que contenga únicamente los nodos $\in X \cup Y \cup Z$. \\ \\
Con esto tenemos una complejidad temporal de \textbf{2n}, siendo \textbf{n} el número de nodos del grafo. Pero, en cada iteración del bucle, se obtienen los predecesores del nodo que se está explorando. El coste de esta acción es $O(p)$, siendo \textbf{p} el número máximo de predecesores de un nodo. Por lo tanto, $p \le m$. En el caso $p = m$, habrá un solo nodo con predecesores, por lo que este nodo tendrá $n - 1$ predecesores, y la complejidad será de $O(2n*m)$, pero rara vez se dará este caso. \\ \\
Así, la complejidad de eliminar los nodos hoja es la siguiente:
$$O(2n*p) = O(2n*p); \; p \le m$$
donde \textbf{n} es el número de nodos del grafo y \textbf{p} es el número máximo de predecesores de un nodo.

\section{Complejidad del paso 2}

\section{Complejidad del paso 3}





\end{document}
